\documentclass{article}
\usepackage[utf8]{inputenc}

\title{Presentation Outline }
\author{Nicholas Alexeev }
\date{\today}

\usepackage{natbib}
\usepackage{graphicx}
\usepackage{listings}
\usepackage{xcolor}

\definecolor{codegreen}{rgb}{0,0.6,0}
\definecolor{codegray}{rgb}{0.5,0.5,0.5}
\definecolor{codepurple}{rgb}{0.58,0,0.82}
\definecolor{backcolour}{rgb}{0.95,0.95,0.92}

\lstdefinestyle{mystyle}{
    backgroundcolor=\color{backcolour},   
    commentstyle=\color{codegreen},
    keywordstyle=\color{magenta},
    numberstyle=\tiny\color{codegray},
    stringstyle=\color{codepurple},
    basicstyle=\ttfamily\footnotesize,
    breakatwhitespace=false,         
    breaklines=true,                 
    captionpos=b,                    
    keepspaces=true,                 
    numbers=left,                    
    numbersep=5pt,                  
    showspaces=false,                
    showstringspaces=false,
    showtabs=false,                  
    tabsize=2
}
\lstset{style=mystyle}
\begin{document}

\maketitle


\section{Presentation}
I have chosen to examine the process by which Oracle relicensed one of its major products, Solaris. 
This issue is interesting because it has had major contributions from third parties 
with the understanding that their work would always be available for free. This issue
is a conflict between uttilitarianism, where Oracle had a short term interest in 
limiting access to their product and maintaining good will with the development community.
\end{document}
